\documentclass[12pt]{article}
\usepackage[spanish]{babel}
\usepackage[utf8]{inputenc}
\usepackage[T1]{fontenc}
\usepackage{vmargin}
\usepackage{setspace}
\usepackage{enumerate}
\usepackage{graphicx}
\usepackage{parskip}
\graphicspath{ {images/} }
\usepackage{float} 

\begin{document}
\begin{center}
\includegraphics[scale=0.5]{unison-logo.png}
\\
\vspace{0.5cm}
UNIVERSIDAD DE SONORA \\
\vspace{0.5cm}
DIVISIÓN DE CIENCIAS EXACTAS Y NATURALES \\
\vspace{0.5cm}
DEPARTAMENTO DE FÍSICA\\
\vspace{0.5cm}
LICENCIATURA EN FÍSICA\\
\vspace{0.5cm}
FÍSICA COMPUTACIONAL I

\vspace{2 cm}
\hrule
\vspace{1 cm}

{\huge \bfseries {El ciclo de actividad magnética solar. Evaluación 2}}

\vspace{1 cm}
\hrule
\vspace{2 cm}
Martinez López Lizbeth Vanessa \\ 
\vspace{1 cm}
Profesor del curso\\
Dr. Carlos Lizárraga Celaya\\
\vspace{2 cm}
26 Abril del 2017
\end{center}
\pagebreak

\begin{doublespace}
\hrule
\section*{Resumen}
Para la evaluación se realizó una transformada discreta de furier para el archivo de datos que contiene el promedio del número de manchas solares por mes, proporcionado por la nasa y encontrar los modos que cumplían con la frecuencia que se obtiene en un período de 11 años o bien 132 meses.
\vspace{0.6 cm}
\hrule

\vspace{0.6 cm}


\section*{Respuesta a las preguntas}
\begin{enumerate}
\item \textbf{De los datos proporcionados, utiliza una transformada discreta de Fourier, para encontrar la frecuencia del ciclo principal. Muestra una gráfica con los principales modos encontrados.}
\begin{center}
\includegraphics[height=10.5 cm, width= 15 cm]{modos.png}
\end{center}
\textbf{\textit{Procedimiento para obtener la transformada de Furier}}

En la actividad de evaluación se menciona que el ciclo solar o ciclo de actividad magnética solar es regular con un periodo principal de 11 años aproximadamente, con esto tenemos que la frecuencia en 1/año es 0.09091 aproximadamente y en 1/mes es 0.007576 aproximadamente. Para el caso de esta actividad, utilicé la columna de mes y la de número de manchas. 

Para comenzar, descargué el archivo proporcionado por la nasa y con Emacs hice un archivo separado por comas, de esta manera sería más fácil leerlo con pandas. Ya con el dato preparado, en Python importé las bibliotecas necesarios e hice lectura del archivo para iniciar con la graficación. El archivo contenía una columna que no me servía, y por ello utilicé el comando \textit{del df['Unnamed: 6']} para eliminarla. 

Con el siguiente código, verifiqué si había precencia de celdas vacías en las columnas:
\begin{verbatim}
df.apply(lambda x: sum(x.isnull()), axis=0)
\end{verbatim}

Al ver que no había renglones donde no hubiera datos, continué con la graficación. Vi que el número de datos en mi archivo es 3213 y empleé el siguiente código para realizar la gráfica de transformada de furier:

\begin{verbatim}
from scipy.fftpack import fft, fftfreq, fftshift
# number of signal points
N = 3213
# sample spacing
T = 1
x = df['Mes']
y = df['No. manchas']
yf = fft(y)
xf = fftfreq(N, T)
xf = fftshift(xf)
yplot = fftshift(yf)
import matplotlib.pyplot as plt
plt.plot(xf, 2.0/N * np.abs(yplot))
plt.xlim(0, 0.03)
fig=plt.gcf()
fig.set_size_inches(7,7)
plt.title('Transformada de Furier para datos de manchas solares')

plt.xlabel("Frecuencia (1/mes)")
plt.ylabel("Número de manchas")

plt.grid()
plt.show()
\end{verbatim}

Esto me permitió graficar el número de manchas en el eje y y la frecuencia en el eje x, N es el número de datos y T el 'periodo' o bien, la separación entre los datos. Para poner las etiquetas de los modos encontrados, primero busqué los que tendrían una frecuencia similar a la del periodo principal, para ello usé lo siguiente:

\begin{verbatim}
print(np.where(a[:,]>15))
b= a[a[:,]>15]
b
\end{verbatim}
Con esto encontré los picos cuya amplitud superara el valor de 15, pues a simple vista vi que los picos que me interesaban analizar eran esos. Seleccionó los modos que estaban en un rango cercano a 0.007 de frecuencia. Los modos que cumplían con eso eran 23,24 y 25

\begin{verbatim}
#Número de manchas
A1= 2*np.absolute(yf[23,]/N)
A2= 2*np.absolute(yf[24,]/N)
A3= 2*np.absolute(yf[25,]/N)

#Frecuencia
f1= xf[int(N/2 +23),]
f2= xf[int(N/2 +24),]
f3= xf[int(N/2 +25),]

#Periodo
p1= 1/f1
p2= 1/f2
p3= 1/f3
\end{verbatim}
Con esta parte de código hice las operaciones necesarias para obtener las amplitudes, frecuencias y períodos, para de esta manera verificar que efectivamente tenían una frecuencia cercana a 0.007 y un período cercano a 11 años. Ya con los modos ubicados, volví a graficar etiquetando. Utilicé el mismo código para graficar, pero agregando lo siguiente:

\begin{verbatim}
plt.text(0.007,41,"1")
plt.text(0.0077,38,"2")
plt.text(.0083,32,"3")
\end{verbatim}

\item \textbf{¿Encuentras un solo ciclo principal o un conjunto de ciclos con frecuencia cercana? ¿Cuál sería el promedio del conjunto de frecuencias?} Un conjunto de ciclos con frecuencia cercana, el promedio del conjunto de frecuencias es: 0.00746965452848 1/mes.

\pagebreak

\item \textbf{¿Que otros ciclos relevantes encuentras? Proporciona una tabla con las amplitudes de los ciclos.}

\begin{table}[ht!]
\centering
\caption{Otros ciclos y sus amplitudes}
\label{my-label}
\begin{tabular}{|l|l|l|}
\hline
Otros modos & Frecuencia (1/mes) & Amplitud      \\ \hline
1           & 0.00093370681606   & 22.7991836131 \\ \hline
2           & 0.00124494242141   & 15.2828121366 \\ \hline
3           & 0.00155617802677   & 13.7359051105 \\ \hline
4           & 0.00591347650171   & 13.0648949851 \\ \hline
5           & 0.00809212573918   & 13.5629086055 \\ \hline
6           & 0.00840336134454   & 30.9080378809 \\ \hline
7           & 0.00902583255524   & 14.1215419264 \\ \hline
\end{tabular}
\end{table}

\item \textbf{Lo que han encontrado hasta ahora son ciertas regularidades, incluso hay pronósticos de un rango para el número de manchas solares. ¿Cómo crees que es posible predecir el número de manchas?} Con un análisis de transformada de furier y haciendo lo de la actividad 7, es decir reconstruir la gráfica para el promedio mensual de las manchas mensuales, usando funciones de senos y cosenos, podemos observar el error relativo y así determinar que tan buena es la aproximación o predicción del número de manchas.

\end{enumerate}
 
    
     
    

 

\end{doublespace}

\end{document}
