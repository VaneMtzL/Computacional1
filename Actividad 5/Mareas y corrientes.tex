\documentclass[12pt]{article}
\usepackage[spanish]{babel}
\usepackage[utf8]{inputenc}
\usepackage[T1]{fontenc}
\usepackage{vmargin}
\usepackage{setspace}
\usepackage{enumerate}
\usepackage{graphicx}
\usepackage{parskip}
\graphicspath{ {images/} }
\usepackage{float} 

\begin{document}
\begin{center}
\includegraphics[scale=0.5]{unison-logo.png}
\\
\vspace{0.5cm}
UNIVERSIDAD DE SONORA \\
\vspace{0.5cm}
DIVISIÓN DE CIENCIAS EXACTAS Y NATURALES \\
\vspace{0.5cm}
DEPARTAMENTO DE FÍSICA\\
\vspace{0.5cm}
LICENCIATURA EN FÍSICA\\
\vspace{0.5cm}
FÍSICA COMPUTACIONAL I

\vspace{2 cm}
\hrule
\vspace{1 cm}

{\huge \bfseries {Mareas y corrientes}}

\vspace{1 cm}
\hrule
\vspace{2 cm}
Martinez López Lizbeth Vanessa \\ 
\vspace{1 cm}
Profesor del curso\\
Dr. Carlos Lizárraga Celaya\\
\vspace{2 cm}
08 Marzo del 2017
\end{center}
\pagebreak

\begin{doublespace}
\hrule
\section*{Resumen}
En el siguiente reporte se habla sobre las mareas y cómo son un fenómeno bastante interesante, las cuales llevan de por medio la física, pues pueden ser explicadas mediante términos que pertenecen a dicha ciencia. También se muestran gráficas sobre el nivel del mar en dos costas a elegir, en este caso se seleccionó Cabo San Lucas, Baja California Sur y Atlantic City, New Jersey. Los datos del nivel del mar fueron obtenidos del sitio web NOAA y CICESE.
\vspace{0.6 cm}
\hrule

\vspace{0.6 cm}

\section*{Introducción}

El sitio web del NOAA menciona que las mareas son uno de los fenómenos más fiables del mundo, pues cuando vemos que el sol sale del este o las estrellas se alzan en la noche, estamos seguros que el nivel del mar suben y bajan a lo largo de las costas.\\ 

Básicamente, las mareas son ondas de muy largo período que se mueven a través de los océanos en respuesta a las fuerzas ejercidas por la luna y el sol. Las mareas se originan en los océanos y avanzan hacia las costas donde aparecen como el ascenso y la caída regular de la superficie del mar.\\ 

Las mareas son un fenómeno bastante interesante, además son importantes en ciertos ámbitos, lo cual las hace un campo bastante amplio de estudio. En este resumen se hablará un poco más a detalle sobre este fenómeno, tocando ciertos puntos interesantes e importantes para el conocimiento de las mareas. 

\section*{Mareas}

Las mareas son el aumento y la caída de los niveles del mar causados por los efectos combinados de las fuerzas gravitacionales ejercidas por la Luna y el Sol y la rotación de la Tierra. Los tiempos y la amplitud de las mareas en cualquier localidad determinada están influenciados por la alineación del Sol y la Luna, por el patrón de las mareas en el océano profundo, por los sistemas anfidrómicos de los océanos y la forma de la costa y la batimetría cercana a la costa.\\

Las mareas varían en escalas de tiempo que van desde horas a años debido a una serie de factores. Para hacer registros precisos, los medidores de mareas en estaciones fijas miden el nivel de agua a lo largo del tiempo.\\ 

Los fenómenos mareales no se limitan a los océanos, sino que pueden ocurrir en otros sistemas siempre que exista un campo gravitacional que varíe en tiempo y espacio.\\
\begin{center}
\includegraphics[scale=0.5]{marea.jpg}

Figura 1. Ejemplo de mareas. 
\end{center}

\section*{Características de las mareas}
Los cambios de marea se realizan a través de las siguientes etapas:

\begin{itemize}
\item El nivel del mar se eleva durante varias horas, cubriendo la zona intermareal.
\item El agua sube a su nivel más alto, alcanzando la marea alta.
\item El nivel del mar cae durante varias horas, revelando la zona intermareal; reflujo.
\item El agua deja de caer, alcanzando la marea baja. 
\end{itemize}

Las mareas son comúnmente semi-diurnas (dos aguas altas y dos aguas bajas cada día), o diurnas (un ciclo de marea por día).

\section*{Componentes de las mareas}
Las componentes de las mareas son el resultado de múltiples influencias que afectan los cambios de las mareas durante ciertos periodos de tiempo. Los componentes primarios incluyen la rotación de la Tierra, la posición de la Luna y el Sol en relación con la Tierra, la altitud de la Luna (elevación) sobre el ecuador de la Tierra y la batimetría. Las variaciones con períodos de menos de medio día se llaman constituyentes armónicos. Por el contrario, los ciclos de días, meses o años se denominan constituyentes de largo período.

\subsection*{Principales componentes lunares semidiurnos}
En la mayoría de las localidades, el constituyente más grande es el "semi-diurno lunar principal", también conocido como el constituyente de marea M2 (o M2). Su período es de aproximadamente 12 horas y 25.2 minutos.\\

Cuando hay dos mareas altas cada día con diferentes alturas (y dos mareas bajas también de diferentes alturas), el patrón se denomina marea semi-diurna mixta.

\subsection*{Rango de variación: Mareas muertas y mareas de primavera}
El rango semi-diurno (la diferencia de altura entre las aguas altas y bajas durante aproximadamente medio día) varía en un ciclo de dos semanas. Dos veces al mes, entre Luna llena y Luna nueva, la Tierra, el Sol y la Luna se alinean; el rango de la marea está entonces en su máximo; ésto se llama la marea de la primavera.\\

Cuando la luna está en primer cuarto y en tercer cuarto, el Sol y la Luna están separados 90$^\circ$ vistos desde la Tierra y la fuerza de marea solar cancela parcialmente la fuerza de marea lunar, en este caso el rango de la marea se encuentra en su mínimo; ésto se llama marea muerta.

\subsection*{Altitud lunar}
La distancia cambiante que separa la Luna y la Tierra también afecta las alturas de las mareas. Cuando la Luna está más cerca, en el perigeo, el rango aumenta, y cuando está en el apogeo, el rango se encoge.

\subsection*{Amplitud y fase}
La etapa o fase de una marea, denotada por el tiempo en horas después del alto nivel de agua, es un concepto útil. La etapa de marea también se mide en grados, con 360$^\circ$ por ciclo de marea. Las líneas de fase constante de marea se llaman líneas cotidal, que son análogas a líneas de contorno de altitud constante en mapas topográficos. El agua alta se alcanza simultáneamente a lo largo de las líneas cotidalas que se extienden desde la costa hasta el océano, y las líneas cotidalas (y por lo tanto las fases de marea) avanzan a lo largo de la costa. Los componentes semi-diurnos y de fase larga se miden desde aguas altas, diurnas desde marea máxima de inundación. \\

Para un océano en forma de una cuenca circular rodeada por un litoral, las líneas cotidalas apuntan radialmente hacia el interior y finalmente se encuentran en un punto común, el punto anfidrómico. El punto anfidrómico es a la vez cotidal con las aguas altas y bajas, que se satisface con el movimiento cero de la marea. 

\section*{Física de las mareas}
La investigación sobre la física de las mareas fue importante en el desarrollo temprano del heliocentrismo y la mecánica celeste, con la existencia de dos mareas diarias explicadas por la gravedad de la Luna. Más tarde, las mareas diarias fueron explicadas más precisamente por la interacción de la Luna y la gravedad del sol.

\subsection*{Fuerzas}
La fuerza de marea producida por un objeto masivo (Luna, en adelante) sobre una pequeña partícula situada sobre o en un cuerpo extenso (Tierra, en adelante) es la diferencia vectorial entre la fuerza gravitacional ejercida por la Luna sobre la partícula y la fuerza gravitatoria que se ejercería sobre la partícula si estuviera situada en el centro de masa de la Tierra. La fuerza gravitacional solar en la Tierra es en promedio 179 veces más fuerte que la lunar, pero debido a que el Sol está en promedio 389 veces más lejos de la Tierra, su gradiente de campo es más débil. La fuerza de marea solar es 46\% tan grande como el lunar.

\subsection*{Amplitud y tiempo de ciclo}
La amplitud teórica de las mareas oceánicas causada por la luna es de unos 54 centímetros (21 pulgadas) en el punto más alto. El sol también causa mareas, de las cuales la amplitud teórica es de unos 25 centímetros (el 46\% de la de la luna) con un tiempo de ciclo de 12 horas. Dado que las órbitas de la Tierra alrededor del Sol y la Luna alrededor de la Tierra son elípticas, las amplitudes de las mareas cambian algo como resultado de las diferentes distancias Tierra-Sol y Tierra-Luna. Esto causa una variación en la fuerza de marea y amplitud teórica de aproximadamente  $\pm$ 18\% para la luna y $\pm$ 5\% para el sol.

\subsection*{Disipación}
Las oscilaciones de las mareas de la Tierra introducen la disipación a una velocidad promedio de unos 3,75 terawatts. Alrededor del 98\% de esta disipación es por mareas marinas. La disipación surge cuando los flujos de marea a escala de cuenca fluyen a flujos de menor escala que experimentan una disipación turbulenta.

\section*{Navegación}
Los flujos de marea son importantes para la navegación, y errores significativos en la posición ocurren si no se acomodan. Hasta el advenimiento de la navegación automatizada, la competencia para calcular los efectos de las mareas era importante para los oficiales navales.

\section*{Otras mareas}
\begin{itemize}
\item \textbf{Mareas de lago:} Grandes lagos como Superior y Erie pueden experimentar mareas de 1 a 4 cm, pero éstas pueden ser enmascaradas por fenómenos meteorológicos inducidos como el seiche.
\item \textbf{Mareas atmosféricas:} Las mareas atmosféricas son de origen gravitacional y térmico y son la dinámica dominante de 80 a 120 kilómetros, por encima de la cual la densidad molecular se vuelve demasiado baja para soportar el comportamiento del fluido.
\item \textbf{Mareas de tierra:} las mareas terrestres afectan a toda la masa terrestre, que actúa de forma similar a un giroscopio líquido con una corteza muy delgada. La corteza terrestre cambia en respuesta a la gravitación lunar y solar, las mareas oceánicas y la carga atmosférica. 
\item \textbf{Mareas galácticas:} Las mareas galácticas son las fuerzas de las mareas ejercidas por las galaxias sobre las estrellas dentro de ellas y las galaxias satelitales orbitándolas.
\end{itemize}

\section*{10 principales componentes armónicos de las mareas}

\begin{table}[H]
\begin{center}


\label{my-label}
\begin{tabular}{|l|l|l|}
\hline
Nombre                                                         &  Símbolo & Periodo (Hrs)  \\ \hline
Límite de agua superficial de la luna principal                &  M$_4$   &  6.210300601   \\ \hline
Límite de agua superficial de la luna principal                &  M$_6$   &  4.140200401   \\ \hline
Agua superficial terdiurnal                                    &  MK$_3$  &  8.177140247   \\ \hline
Abundancia de agua poco profunda de la energía solar principal &  S$_4$   &  6             \\ \hline
Cuarto de agua poco profunda diurna                            &  MN$_4$  &  6.269173724   \\ \hline
Principal lunar semidiurno                                     &  M$_2$   &  12.4206012    \\ \hline
Principal solar semidiurno                                     &  S$_2$   &  12            \\ \hline
Gran lunar elíptico semidiurno                                 &  N$_2$   &  12.65834751   \\ \hline
Lunar diurno                                                   &  K$_1$   &  23.93447213   \\ \hline
Lunar diurno                                                   &  O$_1$   &  25.81933871   \\ \hline
\end{tabular}
\end{center}
\end{table}

\section*{Mareas en Atlantic City y Cabo San Lucas}

Las organizaciones como NOAA y CICESE se encargan de recopilar información sobre las mareas en distintas costas, CICESE es la organización para costas mexicanas, mientras que NOAA es la organización que se encarga de las costas estadounidenses. La recopilación de esta información, permite hacer análisis del comportamiento de las mareas, para así poder hacer predicciones sobre fenómenos meteorológicos, entre otras cosas. 

En esta práctica hicimos uso de archivos obtenidos de dichas organizaciones, para ver el comportamiento de las mareas de ciertas costas (en este caso Cabo San Lucas y Atlantic City), en un determinado mes a elegir. Por medio de gráficos pudimos mostrar dicho comportamiento. 

\subsection*{Gráficas}

\textbf{Atlantic City, New Jersey}

Para Atlantic City se seleccionó el mes de Enero del 2017, a continuación se presenta la gráfica obtenida por medio de Python:

\begin{center}
\includegraphics[scale=0.8]{grafica1}

\textit{Figura 2. Gráfica de nivel del agua vs fecha. Comportamiendo de mareas, Atlantic City. Enero 2017.}
\end{center}

\textbf{Cabo San Lucas, Baja California Sur}

Para Cabo San Lucas se seleccionó el mes de Abril del 2016, pues en la página oficial del CICESE se pueden descargar los archivos anuales para el año 2016. A continuación se presenta la gráfica obtenida por medio de Python:

\begin{center}
\includegraphics[scale=0.8]{grafica2}

\textit{Figura 3. Gráfica de nivel del agua vs fecha. Comportamiendo de mareas, Cabo San Lucas. Abril 2016.}
\end{center}

Como ya se mencionó, estas gráficas representan el comportamiento de las mareas en un período de tiempo, en este caso el período seleccionado fue de un mes.


\section*{Conclusión}
Las mareas son fenómenos bastante interesantes con mucha teoría de por medio. La práctica no pretendía volvernos unos expertos en el tema, pero sí dejarnos buenos conocimientos sobre las mareas, para así poder tener un mejor entendimiento de las ya mencionadas. Como se pudo ver, las mareas son fenómenos un tanto complejos y que tienen un gran campo de estudio, pues es necesario analizarlas para lograr predicciones de otros fenómenos, además de poder determinar las condiciones en cuanto a la navegación y otras cosas.

\newpage
\renewcommand{\refname}{\section*{Referencias}}
\begin{thebibliography}{9}
\bibitem{a1} \textsc{$https://en.wikipedia.org/wiki/Tide$}

\bibitem{b1} \textsc{$https://en.wikipedia.org/wiki/Theory\_of\_tides$}

\bibitem{c1} \textsc{$https://tidesandcurrents.noaa.gov/stations.html?type=Water+Levels\%29$}

\bibitem{d1} \textsc{$http://predmar.cicese.mx/$}

\bibitem{m2} \textit{Figura 1.} \textsc{$http://www.tablademareas.com/mareas/coeficiente-marea$}
\end{thebibliography}
\end{doublespace}
\end{document}
