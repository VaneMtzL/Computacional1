\documentclass{article}
\usepackage[spanish]{babel}
\usepackage[utf8]{inputenc}
\usepackage[T1]{fontenc}
\usepackage{vmargin}
\usepackage{setspace}
\usepackage{enumerate}
\usepackage{graphicx}
\graphicspath{ {images/} }
\usepackage{float} 

\begin{document}
\begin{center}
\includegraphics[scale=0.5]{Escudo_Unison.png}
\\
\vspace{0.5cm}
UNIVERSIDAD DE SONORA \\
\vspace{0.5cm}
DIVISIÓN DE CIENCIAS EXACTAS Y NATURALES \\
\vspace{0.5cm}
DEPARTAMENTO DE FÍSICA\\
\vspace{0.5cm}
LICENCIATURA EN FÍSICA\\
\vspace{0.5cm}
FÍSICA COMPUTACIONAL I

\vspace{2 cm}
\hrule
\vspace{1 cm}

{\huge \bfseries {La Atmósfera Terrestre}}
\\

\vspace{1 cm}
\hrule
\vspace{2 cm}
Martinez López Lizbeth Vanessa \\ 
\vspace{1 cm}
Profesor del curso\\
Dr. Carlos Lizárraga\\
\vspace{2 cm}
30 de Enero del 2017
\end{center}


\pagebreak
\begin{doublespace}

\hrule
\section*{Resumen}
En el siguiente texto se definirá lo que es atmósfera de manera general, además se hablará acerca de la estructura y propiedades de la atmósfera terrestre. Se ha dado más énfasis en las capas que conforma la atmósfera, así como en las características principales. El objetivo es dar a conocer más acerca de este tema que se aborda en ciertas asignaturas escolares, mas no a profundidad.
\vspace{0.5 cm}
\hrule

\vspace{0.6 cm}
\section{Introducción}
En algún momento de nuestras vidas, por no decir cada día, hemos visto hacia el cielo, lo hemos admirado en aquellos días soleados en el que no hay nubes, así como en aquellos días en los que sí hay, lo hemos visto en la noche y tal vez sólo hemos admirado las estrellas y la Luna, aquellos cuerpos celestes que están muy lejos de nosotros, pero ¿nos hemos preguntado qué hay más cerca de nosotros?, ¿qué es lo que rodea a nuestro planeta y no podemos ver?, o bien, ¿por qué el cielo es azul?, o ¿qué nos protege del peligro que traen consigo los rayos ultravioleta? 
\\

Atmósfera es una palabra que tal vez todos conocemos y muy pocos entendendemos con exactitud qué es. Es cierto que en el transcurso de nuestros estudios llega un momento en el que se habla sobre la atmósfera y se explica lo más básico, como de qué se compone y cuáles son sus capas, sin embargo no se profundiza más y esto nos lleva a desconocer el porqué de muchas cosas. Es por eso que la finalidad de este texto es profundizar un poco más en este tema y de esta forma dar respuesta a ciertas dudas. Primero debemos iniciar por definir atmósfera.

\section{La atmósfera y sus características}
Una atmósfera es una capa compuesta por gases la cual rodea un planeta o cualquier otro cuerpo material. La atmósfera se mantiene en su lugar, rodeando el planeta o cuerpo, debido a la gravedad de dicho cuerpo material, cabe destacar que es más probable retener a la atmósfera si su temperatura es muy baja y la gravedad a la que se somete es alta. Por ejemplo, en la atmósfera de nuestro planeta, las moléculas que la componen son arrastradas cerca de la superficie terrestre por la gravedad, haciendo que la atmósfera se concentre en la superficie de la Tierra y se adelgace rápidamente con la altura.
\\

La atmósfera del planeta Tierra está compuesta por un 78\% de nitrógeno,  21\% oxígeno y el 1\% de otros gases, como el argón. El espesor de esta capa de gases invisible es de aproximadamente 10000 km. Ciertas característica sobresaliente de la atmósfera es que la presión disminuye conforme la altura aumenta. La presión del aire se puede explicar cómo la medida del peso de las moléculas de gases por encima de nosotros y debido a que las moléculas se concentran en la superficie terrestre, conforme aumenta la altura, menos moléculas hay sobre nosotros. Al igual que la presión, la densidad disminuye mientras la altura aumenta. Por otro lado, la temperatura tiende a disminuir con la altura, sin embargo es irregular, pues en algunas zonas altas de la atmósfera, la temperatura aumenta.
\\

Como hemos notado desde hace mucho tiempo, en el día el cielo se ve de un color azul, esto se debe a que la luz del Sol, que se compone de varios colores, es dispersada por las  moléculas de aire, de manera que a nuestros ojos llega principalmente el azul. En cambio, al atardecer o en el amanecer los rayos inciden de forma oblicua en la Tierra, realizan un mayor recorrido hasta alcanzar la superficie terrestre. Durante este camino se absorben todos los colores y sólo llegan los rojizos.
\\

La atmósfera es una parte importante que hace que la Tierra sea habitable, pues el oxígeno que hay dentro de la atmósfera es esencial para la vida, además bloquea los rayos peligrosos provenientes del Sol, así estos no llegan a la Tierra y atrapa el calor para que tengamos una temperatura cómoda. En la atmósfera se producen todos los fenómenos climáticos y meteorológicos que afectan al planeta; regula la entrada y salida de energía de la tierra y es el principal medio de transferencia del calor.

\section{Estructura de la atmósfera}
Generalmente se menciona que la atmósfera está compuesta por cuatro capas, en otros libros, artículos o referencias, se mencionan cinco, cada una de estas capas tienen sus características, desde grosor, composición, temperatura, procesos que se llevan a cabo dentro de ellas, etcétera. Estas cinco capas son: troposfera, estratosfera, mesosfera, termosfera o ionosfera y Exosfera. Las divisiones entre una capa y otra se denominan respectivamente tropopausa, estratopausa, mesopausa y termopausa. Las distintas variaciones de la temperatura, ayudan a definir las distintas zonas de la atmósfera

\begin{center} \includegraphics[scale=0.28]{capas_atm.png} 
\\Imagen 1. Capas de la atmósfera y temperatura. \end{center} 
 

\subsection{Troposfera}
La troposfera es la capa que se encuentra en íntimo contacto con la superficie terrestre, esta capa contiene la mitad de la atmósfera terrestre. La altitud de la troposfera está entre los 8 y 20 km. Esta parte de la atmósfera es la más densa, pues aquí se concentra la mayor parte del oxígeno y vapor de agua. El vapor de agua permite regular la temperatura. Cabe destacar, que en esta capa la temperatura decrece conforme la altura aumenta, en un aproximado de 6.5 $^{\circ}$C por cada kilómetro que se asciende. 
\\

En la troposfera se llevan a cabo todos los fenómenos meteorológicos, como la lluvia, huracanes o vientos.

\subsection{Estratosfera}
La estratosfera inicia justo después de la troposfera, se extiende aproximadamente 50 km de altura. Aquí se encuentra lo que conocemos como capa de ozono, dicha capa se encarga de absorver y dispersar las radiaciones nocivas que llegan a la Tierra, permitiéndole el pase únicamente a las radiaciones que permiten la vida en el planeta. 
\\

En esta capa, la temperatura asciende conforme la altura aumenta, esto se debe a la absorción de los rayos ultravioleta y la presencia de ozono.Este perfil de temperaturas permite que la capa sea muy estable y evita turbulencias, es por eso que muchos aviones de reacción prefieren volar en la estratosfera.

\subsection{Mesosfera}

La mesosfera se extiene hasta aproximadamente 80 km de altura, la temperatura disminuye con el aumento de la altura, llegando casi a los -100 $^{\circ}$C, siendo así la zona más fría de la atmósfera terrestre. La baja densidad del aire en la mesosfera determina la formación de turbulencias.
\\

En la mesosfera se observan las caídas de meteoritos (ya desintegrados), los cuales emiten luz debido a la fricción por entrar en contacto con esta capa, siendo esto lo que conocemos como estrellas fugaces.

\subsection{Termosfera}
La termosfera inicia justo después de la mesosfera y se extiende hasta 600 km de altura, aquí la temperatura aumenta hasta aproximadamente 1000 $^{\circ}$C, este gran aumento de temperatura se debe a la ionización de los gases, provocado por los rayos ultravioleta, gamma y X. 
\\

En esta capa ocurre la desintegración de los meteoritos. En las regiones polares las partículas cargadas portadas por el viento solar son atrapadas por el campo magnético terrestre, dando lugar a otro fenómeno que ocurre dentro de la termosfera: la formación de las auroras; además, es en la termosfera donde los satélites orbitan.
\subsection{Exosfera}
La exosfera es la capa límite superior de la atmósfera terrestre, se extiende desde donde finaliza la termosfera, hasta los 10,000 km. La exosfera es la zona de tránsito entre la atmósfera terrestre y el espacio. Es una capa bastante delgada que termina fundiéndose con el espacio, es decir, los gases pierden sus propiedades físico-químicas y poco a poco se dispersan hasta que la composición de esta capa termina siendo similar a la composición del espacio. En esta capa hay un alto contenido de polvo cósmico.

\section{Globo meteorológico}
Una pregunta que puede rondar por nuestras mentes es: ¿cómo se determinaron las características y/o propiedades de la atmósfera? Tenemos claro que seguramente hubo muchas investigaciones y tal vez en algún momento no se contaba con las herramientas adecuadas para determinar esta información, sin embargo es importante decir que desde 1896 se hace uso del globo meteorológico. 
\\

El globo meteorológico es un globo aerostático compuesto de látex de gran flexibilidad. Este globo eleva instrumentos a la atmósfera para obtener información sobre la presión, humedad, temperatura y velocidad del viento en las zonas de la atmósfera, esta información es suministrada por medio de un aparato pequeño de medición llamado radiosonda, dicho aparato es desechable. Para obtener los datos del viento, pueden ser rastreados por radar, radiolocalización o bien por sistemas de navegación (como el GPS).
\\

Una de las primeras personas en usar el globo meteorológico, fue el meteorólogo francés Léon Teisserenc de Bort, que a partir de 1896 lanzó cientos de globos meteorológicos desde su observatorio en Trappes, Francia, esto llevó al descubrimiento de la tropopausa y la estratosfera.

\section{Conclusión}
La atmósfera, esa gran "burbuja" que nos rodea y no podemos ver, es de suma importancia para la existencia de vida en el planeta Tierra; además es, junto a otros factores, la causante de muchos de los procesos o fenómenos que vemos ocurrir prácticamente a diario o en algún momento de nuestras vidas.
\\

La atmósfera, a pesar de ser algo que podemos mencionar con facilidad y creer entender, es mucho más compleja de lo que parece, profundizar un poco más en este tema permite obtener respuesta a todas esas preguntas que en algún momento se formulan en nuestras mentes. Es un tema bastante extenso, pero interesante y si no se le hubiese dado la importancia necesaria, tal vez ignoraríamos muchas cosas importantes y beneficios que nos brinda la atmósfera terrestre. 
\\

\section*{Referencias}
\begin{itemize}
    \item https://en.wikipedia.org/wiki/Atmosphere Consultada el 28 de Enero del 2017.
    \item http://climate.ncsu.edu/edu/k12/.AtmStructure Consultada el 26 de Enero del 2017.
    \item	http://www.windows2universe.org/earth/Atmosphere/overview.html Consultada el 29 
    \\de Enero del 2017.
    \item	$http://recursostic.educacion.es/secundaria/edad/1esobiologia/1quincena5/1q5_centro.htm$ Consultada el 29 de Enero del 2017.
    \item $http://www7.uc.cl/sw_educ/contam/fratmosf.htm$ Consultada el 29 de Enero del 2017.
    \item http://cambioclimaticoglobal.com/atmosfe1 Consultada el 30 de Enero del 2017.
    \item $https://www.nasa.gov/mission_pages/sunearth/science/atmosphere-layers2.html$ Consultada el 29 de Enero del 2017.
    \item $https://en.wikipedia.org/wiki/Weather_balloon$ Consultada el 27 de Enero del 2017.
    \item Imagen 1. $http://recursostic.educacion.es/secundaria/edad/1esobiologia/1quincena5
\\/1q5_contenidos_1e.htm$
  
    
\end{itemize}


\end{doublespace}

\end{document}
