\documentclass{article}
\usepackage[spanish]{babel}
\usepackage[utf8]{inputenc}
\usepackage[T1]{fontenc}
\usepackage{vmargin}
\usepackage{setspace}
\usepackage{enumerate}
\usepackage{graphicx}
\graphicspath{ {images/} }
\usepackage{float} 

\begin{document}
\begin{center}
\includegraphics[scale=0.5]{Escudo_Unison.png}
\\
\vspace{0.5cm}
UNIVERSIDAD DE SONORA \\
\vspace{0.5cm}
DIVISIÓN DE CIENCIAS EXACTAS Y NATURALES \\
\vspace{0.5cm}
DEPARTAMENTO DE FÍSICA\\
\vspace{0.5cm}
LICENCIATURA EN FÍSICA\\
\vspace{0.5cm}
FÍSICA COMPUTACIONAL I

\vspace{2 cm}
\hrule
\vspace{1 cm}

{\huge \bfseries {Preguntas de reflexión}}
\\

\vspace{1 cm}
\hrule
\vspace{2 cm}
Martinez López Lizbeth Vanessa \\ 
\vspace{1 cm}
Profesor del curso\\
Dr. Carlos Lizárraga\\
\vspace{2 cm}
30 de Enero del 2017
\end{center}
\pagebreak

\begin{center}
\section*{Preguntas de reflexión}
\end{center}
\begin{itemize}
{\item \bfseries ¿Cuál es tu primera impresión del uso de LaTeX?}
\\
Mi primera impresión fue que LaTeX es bastante complejo, pues hay demasiados formatos y se debe ser cuidadoso en cómo se usan, además de la sintaxis, sin embargo ya que logras comprenderlo, se vuelve manejable y fácil.
{\item \bfseries ¿Qué aspectos te gustaron más?}
\\
La manera en la que todo queda bien organizado, dando un formato que es adecuado para nosotros, además tiene funciones que son de bastante ayuda al momento de escribir nuestros trabajos.
{\item \bfseries ¿Qué no pudiste hacer en LaTeX?}
\\
Manejar bien la bibliografía y quitar la numeración de página de la portada, pues esta no debe ir numerada.
{\item \bfseries En tu experiencia, comparado con otros editores, ¿cómo se compara LaTex?}
\\
Claramente es un poco más complejo que otros editores que he usado, debido a los comandos, sin embargo eso es lo de menos, pues a mi parecer LaTeX es mucho mejor para crear textos científicos, tiene todo lo que necesitamos y el formato es mucho mejor.
{\item \bfseries ¿Qué es lo que más te llamó la atención en el desarrollo de esta actividad?}
\\
Aprender a utilizar LaTeX, antes de entrar a la carrera de Licenciatura en Física, no tenía ni idea de que existían editores de texto como LaTeX, en el cual se utilizaban comandos, y eso ha llamo mucho mi atención.
{\item \bfseries ¿Qué cambiarías en esta actividad?}
\\
Que se especificara exactamente el tipo de texto a escribir, es decir: ensayo, resumen, síntesis, monografía, etc.
{\item \bfseries ¿Qué consideras que hace falta en esta actividad?}
\\
En mi opinión, no hace falta nada, me pareció bastante bien.
{\item \bfseries ¿Puedes compartir alguna referencia nueva que consideras útil y no se haya contemplado?}
\\ Sí, en lo particular me gustó la información proporcionada en estas páginas:	
\\
$http://recursostic.educacion.es/secundaria/edad/1esobiologia/1quincena5/1q5_centro.htm$
\\
$https://www.nasa.gov/mission_pages/sunearth/science/atmosphere-layers2.html$
\\
$http://www.windows2universe.org/earth/Atmosphere/overview.html$
{\item \bfseries ¿Algún comentario adicional que desees compartir?}
\\
Solamente que me gustó bastante la actividad, pues de esta forma pude conocer y aprender a utilizar LaTeX, un editor de texto que me será bastante útil de hoy en adelante.
\end{itemize}
\end{document}
