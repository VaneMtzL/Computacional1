\documentclass[12pt]{article}
\usepackage[spanish]{babel}
\usepackage[utf8]{inputenc}
\usepackage[T1]{fontenc}
\usepackage{vmargin}
\usepackage{setspace}
\usepackage{enumerate}
\usepackage{graphicx}
\graphicspath{ {images/} }
\usepackage{float} 

\begin{document}
\begin{center}
\includegraphics[scale=0.5]{unison-logo.png}
\\
\vspace{0.5cm}
UNIVERSIDAD DE SONORA \\
\vspace{0.5cm}
DIVISIÓN DE CIENCIAS EXACTAS Y NATURALES \\
\vspace{0.5cm}
DEPARTAMENTO DE FÍSICA\\
\vspace{0.5cm}
LICENCIATURA EN FÍSICA\\
\vspace{0.5cm}
FÍSICA COMPUTACIONAL I

\vspace{2 cm}
\hrule
\vspace{1 cm}

\begin{doublespace}
{\huge \bfseries {Iniciándose con el Editor de Texto 

Gnu Emacs}}
\end{doublespace}

\vspace{1 cm}
\hrule
\vspace{2 cm}
Martinez López Lizbeth Vanessa \\ 
\vspace{1 cm}
Profesor del curso\\
Dr. Carlos Lizárraga\\
\vspace{2 cm}
09 de Febrero del 2017
\end{center}
\pagebreak

\begin{doublespace}

\hrule
\section*{Resumen}
En el siguiente reporte se hablará acerca del editor de textos Gnu Emacs, además se mostrarán los resultados obtenidos al usar por primera vez dicho editor de textos. Los resultados son en base al tema tratado en esta actividad: "Limpieza y Preparación de datos Usando Emacs"; para desarrollar esta actividad se hizo uso de datos atmosféricos obtenidos de la Universidad de Wyoming, haciendo únicamente enfoque en los datos de la estación de Albuquerque, Nuevo México.

\vspace{0.6 cm}
\hrule

\vspace{0.6 cm}

\section{Introducción}
GNU Emacs es un editor de textos, es decir, es un programa que permite editar y/o crear archivos de texto sin formato (también llamado texto plano). Emacs en particular, es un editor bastante popular entre los programadores debido a su gran variedad de funciones.
\\

Una función bastante útil y la cual fue utilizada en esta actividad, es la preparación y limpieza de datos. Cuando se trata de manejar una base de datos de gran cantidad (un archivo que contiene miles de datos, por ejemplo), es poco práctico hacerlo de forma manual, revisando cada columna y cada celda. Emacs se encarga de facilitar el manejo de los archivos haciendo uso de comandos.
\\

Se podría profundizar mucho sobre las funciones de Emacs, sin embargo éste no fue exactamente el objetivo de la actividad realizada. Lo que se pretendía lograr (y lo cual se ha logrado), fue lo siguiente: elegir una estación de la lista proporcionada por el profesor, para así obtener datos recolectados por sondas atmosféricas que son lanzadas varias veces al día, al contar con estos datos, los cuáles son organizados y proporcionados por la universidad de Wyoming, el objetivo era hacer uso de Emacs y así hacer una limpieza de datos, creando archivos con la información necesaria, esto nos permitiría adquirir conocimientos básicos, pero muy útiles, del editor de texto.

\section{Desarrollo}
La actividad 2 inició con la elección de la estación con la que trabajaría cada alumno, para esto el profesor nos proporcionó dos listas, una en la que había estaciones de México y otra con estaciones del sur de Estados Unidos. En mi cado, elegí la estación de Albuquerque, Nuevo México.
\\

Ya con la estación elegida, la actividad procedió con la recolección de los datos de las observaciones de sondeo del día 1 de Febrero del 2017 a las 12Z, además de los datos de todo el año de 2016 a las 00Z y 12Z. La recolección de datos se llevó a cabo por medio de un script proporcionado por el profesor, éste nos ayudó a facilitar la extracción de los datos y guardarlos en archivos con formato de texto. 

\subsection{Datos del 1 de Febrero del 2017}
Al tener el archivo con los datos del 1 de Febrero, se hicieron dos gráficas: presión contra altura y temperatura contra altura. Las gráficas se realizaron con gnuplot y la finalidad de éstas, fue analizar el comportamiento de la presión y la altura respecto al tiempo, además de percatarnos que lo mencionado en la actividad anterior se cumple.
\\

A continuación se muestran las gráficas obtenidas:
\\
\begin{center}
\includegraphics[scale=0.8]{Presion.png}
\vspace{1.5 cm}

\includegraphics[scale=0.8]{Temperatura.png}
\end{center}
\pagebreak

Con las gráficas podemos ver que la presión disminuye exponencialmente con respecto a la altura, ésto se debe a que la densidad del aire es cada vez menor conforme la altura aumenta y como la presión depende de la densidad y la altura, es claro que ésta disminuya. En el caso de la temperatura, se puede observar que varía de una forma diferente según ciertas regiones de la atmósfera, hasta poco más de los 10000 metros de altura la temperatura va disminuyendo, en una pequeña región entre los 10000 y 15000 metros, ésta aumenta levemente, hasta los 20000 metros la temperatura sigue disminuyendo y después de ahí comienza a aumentar con respecto a la altura. Sin embargo el intervalo de temperatura ronda entre temperaturas demasiado pequeñas, pues va de aproximadamente -80 $^{\circ}$C a 10 $^{\circ}$C. 

\subsection{Datos del año 2016}
Teniendo el archivo con los datos de todo el año 2016, se hizo uso del comando grep y wc, los cuales permitían obtener la cantidad de datos que hubo en cada mes. Grep se usa para filtrar palabras, así sólo se imprimen en pantalla o en archivo los renglones que contienen la palabra escrita en el comando grep, y wc muestra la cantidad de renglones o caracteres en un archivo de texto. El comando utilizado fue el siguiente:

\begin{verbatim}
grep Observations sondeos.txt | grep Jun | grep 00Z | wc
\end{verbatim}

Gracias a este comando se pudo saber la cantidad de observaciones que hubo en cada mes y a cierta hora. A continuación se muestra la tabla donde se reflejan dichos datos:

\pagebreak

\begin{center}
\subsection*{Tabla de cantidad de observaciones por mes}
\end{center}
\begin{table}[htbp]
\begin{center}
\begin{tabular}{|l|l|l|}
\hline \hline
Mes & 00Z & 12Z  \\
\hline \hline
Enero & 31 & 31 \\ \hline
Febrero & 28 & 28\\ \hline
Marzo & 31 & 31\\ \hline
Abril & 30 & 30 \\ \hline
Mayo & 31 & 31 \\ \hline
Junio & 30 & 30 \\ \hline
Julio & 31 & 31 \\ \hline
Agosto & 31 & 31 \\ \hline
Septiembre & 30 & 30 \\ \hline
Octubre & 30 & 31 \\ \hline
Noviembre & 30 & 30 \\ \hline
Diciembre & 31 & 31 \\ \hline
Total & 364 & 365 \\ \hline
\end{tabular}
\label{tabla:sencilla}
\end{center}
\end{table}

Según los resultados obtenidos en la tabla, la estación de Albuquerque recopiló los datos de sondeo prácticamente todo el año, solamente les faltó uno a las 00Z en el mes de Octubre.

\subsection{Recopilación de datos}
Para extraer los datos y recopilarlos en un archivo, se hizo uso de un script proporcionado por el profesor. Dicho script fue editado por medio de Emacs, para así poder hacer un ejecutable y obtener los datos requeridos.

\subsection*{Script para la recopilación de datos del 1 de Febrero del 2017}
\begin{verbatim}
# Descarga por mes. Cambiar año de consulta. Ajustar el numero de estacion.
#!/bin/bash
# Despues de editar: chmod 755 script1.sh
# Para ejecutar: ./script1.sh

IFS=":"
LISTM31="01:03:05:07:08:10:12"
#LISTM31="01:03:05:07"
LISTM30="04:06:09:11"
#LISTM30="04:06"
LISTM28="02"

    /usr/bin/wget "http://weather.uwyo.edu/cgi-bin/sounding?region=naconf&
    TYPE=TEXT%3ALIST&YEAR=2017&MONTH=02&FROM=0112&TO=0112&STNM=72365"

       /bin/sleep 5
       
done
\end{verbatim}

\subsection*{Script para la recopilación de datos del año 2016}
\begin{verbatim}
# Descarga por mes. Cambiar año de consulta. Ajustar el numero de estacion.
#!/bin/bash
# Despues de editar: chmod 755 script1.sh
# Para ejecutar: ./script1.sh

IFS=":"
LISTM31="01:03:05:07:08:10:12"
#LISTM31="01:03:05:07"
LISTM30="04:06:09:11"
#LISTM30="04:06"
LISTM28="02"

# Script para bajar info por mes. Año 2016, dentro del URL:  YEAR=2015

# Months 31 days

for i in $LISTM31 ; do

    /usr/bin/wget "http://weather.uwyo.edu/cgi-bin/sounding?region=naconf&
    TYPE=TEXT%3ALIST&YEAR=2016&MONTH=$i&FROM=0100&TO=3112&STNM=72365"

    /bin/sleep 5

done

# Months 30 days

for i in $LISTM30 ; do

    /usr/bin/wget "http://weather.uwyo.edu/cgi-bin/sounding?region=naconf&
    TYPE=TEXT%3ALIST&YEAR=2016&MONTH=$i&FROM=0100&TO=3012&STNM=72365"

    /bin/sleep 5

done

# Feb. 28 days

for i in $LISTM28 ; do

    /usr/bin/wget "http://weather.uwyo.edu/cgi-bin/sounding?region=naconf&
    TYPE=TEXT%3ALIST&YEAR=2016&MONTH=$i&FROM=0100&TO=2812&STNM=72365"

    /bin/sleep 5

    done
\end{verbatim}

Con ayuda de Emacs, usando ''Esc + x'' y escribiendo query-replace, es posible reemplazar cualquier palabra o frase del archivo con lo que requieres. Al editar el script proporcionado por el profesor se hizo uso de esos comandos, así se pudo reemplazar el STNM con el número de la estación de Albuquerque y el año.

\section{Conclusión}
Manejar y organizar un número bastante grande de datos es difícil si se trata de hacer de forma manual, es por ello que conocer editores de texto y saber sus funciones básicas puede marcar la diferencia, ya que al contar con estas herramientas tan interesantes e importantes, se nos facilitan las cosas. En esta práctica se pudo ver que Emacs fue un gran apoyo para el manejo y preparación de datos, además contando con la información recopilada de la Universidad de Wyoming se pudo observar que lo dicho en la práctica anterior es realmente cierto, pues los datos lo demuestran.

\newpage
\renewcommand{\refname}{\section{Referencias}}
\begin{thebibliography}{9}
\bibitem{a1} \textsc{$https://es.wikipedia.org/wiki/Emacs$}

\bibitem{b1} \textsc{$https://es.wikipedia.org/wiki/Editor_de_texto$}

\bibitem{c1} \textsc{$http://weather.uwyo.edu/upperair/sounding.html$}
\end{thebibliography}

\end{doublespace}
\end{document}
