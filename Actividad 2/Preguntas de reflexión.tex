\documentclass[12 pt]{article}
\usepackage[spanish]{babel}
\usepackage[utf8]{inputenc}
\usepackage[T1]{fontenc}
\usepackage{vmargin}
\usepackage{setspace}
\usepackage{enumerate}
\usepackage{graphicx}
\graphicspath{ {images/} }
\usepackage{float} 
\begin{document}
\begin{center}
\includegraphics[scale=0.5]{unison-logo.png}
\\
\vspace{0.5cm}
UNIVERSIDAD DE SONORA \\
\vspace{0.5cm}
DIVISIÓN DE CIENCIAS EXACTAS Y NATURALES \\
\vspace{0.5cm}
DEPARTAMENTO DE FÍSICA\\
\vspace{0.5cm}
LICENCIATURA EN FÍSICA\\
\vspace{0.5cm}
FÍSICA COMPUTACIONAL I

\vspace{2 cm}
\hrule
\vspace{1 cm}

{\huge \bfseries {Preguntas de reflexión}}
\\

\vspace{1 cm}
\hrule
\vspace{2 cm}
Martinez López Lizbeth Vanessa \\ 
\vspace{1 cm}
Profesor del curso\\
Dr. Carlos Lizárraga\\
\vspace{2 cm}
09 de Febrero del 2017
\end{center}
\pagebreak

\begin{center}
\section*{Preguntas de reflexión}
\end{center}
\vspace{1 cm}

\begin{itemize}
{\item \bfseries ¿Cuál es tu primera impresión del uso de bash/Emacs?}
\\
Mi primera impresión fue que Emacs era un editor de texto simple, como muchos otros, sin embargo al utilizarlo más y ver la cantidad de funciones que tiene, me di cuenta que fue una primera impresión bastante errónea.
{\item \bfseries ¿Ya lo habías utilizado?}
\\
No, sabía de su existencia y en algún momento lo llegué a abrir, sin embargo nunca llamó mi atención utilizarlo.

{\item \bfseries ¿Qué cosas se te dificultaron más en bash/Emacs? }
\\
Recordar que ciertos comandos no hacen lo mismo que entras partes, por ejemplo hacer uso de ctrl + z, ctrl + v, y esas cosas que se van corrigiendo conforme te acostumbrar a utilizar Emacs.

{\item \bfseries ¿Qué ventajas le ves a Emacs?}
\\
Pues con esta actividad me di cuenta que una muy buena ventaja es la recopilación de datos, no es necesario ir a la página de la Universidad de Wyoming y copiar cada lista de datos, lo cuál sería bastante tardado y tedioso.

{\item \bfseries ¿Qué es lo que mas te llamó la atención en el desarrollo de esta actividad?}
\\
Crear el script para obtener los datos de todo el año. Usar los comandos para saber cuántas observaciones hubo en cada mes.

{\item \bfseries ¿Qué cambiarías en esta actividad?}
\\
Nada.

{\item \bfseries ¿Qué consideras que falta en esta actividad?}
\\
No creo que le haya hecho falta algo.

{\item \bfseries ¿Puedes compartir alguna referencia nueva que consideras útil y no se haya contemplado?}
\\
En esta ocasión no busqué regerencias nuevas, pues todo lo que necesitaba estaba en las proporcionadas por el profesor o las que utilicé en la actividad anterior.

{\item \bfseries ¿Algún comentario adicional que desees compartir?}
\\
Me gustó bastante la práctica, pues así pude conocer un poco más de Emacs y darme cuenta de por qué es tan popular entre programadores.

\end{itemize}
\end{document}
